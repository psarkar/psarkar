\documentclass[11pt]{article}
\usepackage[left=1.25in,top=1in,right=1.25in,bottom=1.00in]{geometry}
\usepackage{amsmath,amssymb}
\usepackage{amsbsy}
\usepackage{amsthm}
\usepackage{hyperref}

\usepackage{epsfig}
\usepackage{color}
\usepackage[round]{natbib}
\usepackage{multirow} 
\newcommand{\logit}{\mbox{logit}}
\newcommand{\probit}{\mbox{probit}}
\newcommand{\hiw}{{\small\textsc{HIW}}}
\newcommand{\iw}{{\small\textsc{IW}}}
\newcommand{\N}{\mbox{N}}
\newcommand{\Be}{\mbox{Be}}
\newcommand{\dd}{\mbox{d}}
\newcommand{\C}{\; | \;}
\newcommand{\R}{\mathbb{R}}
\newcommand{\var}{\text{var}}
%\newcommand{\R}{\mathcal{R}}
\newcommand{\F}{\mathcal{F}}

\renewcommand{\S}{\mathcal{S}}
\newtheorem{theorem}{Theorem}[section]
\newtheorem{lemma}[theorem]{Lemma}
\newtheorem{proposition}[theorem]{Proposition}
\newtheorem{corollary}[theorem]{Corollary}

\newenvironment{definition}[1][Definition]{\begin{trivlist}
\item[\hskip \labelsep {\bfseries #1}]}{\end{trivlist}}
\newenvironment{example}[1][Example]{\begin{trivlist}
\item[\hskip \labelsep {\bfseries #1}]}{\end{trivlist}}
\newenvironment{remark}[1][Remark]{\begin{trivlist}
\item[\hskip \labelsep {\bfseries #1}]}{\end{trivlist}}




\begin{document}

\title{{\bf Homework Assignment 4}\\Due via Canvas, May 7th by midnight}
\author{SDS 384-11 Theoretical Statistics}

\date{}

\maketitle{}
%\textbf{Set algebra and probability laws.}
\begin{enumerate}%\item Read Bertsekas and Tsitsiklis, sections 1.1, 1.2 and 1.3.
%\section{Jeffrey's prior}
%\begin{enumerate}
%\end{enumerate}

\item Consider an i.i.d. sample of size $n$ from a discrete distribution parametrized by $p_1,\dots, p_{m-1}$  on $m$ atoms. A common test for uniformity of the distribution is to look at the fraction of pairs that collide, or are equal. Call this statistic $U$.
\begin{enumerate}
	%	\item What is the variance of $U$?
	\item Is $U$ a U statistic? When is it degenerate?
	\item What is the variance of $U$? Please give the exact answer, without approximation. 
	\item For a hypothesis test, we will consider alternative distributions which have $p_i=\frac{1+a}{m}$ for half of the atoms in the distribution and $\frac{1-a}{m}$ for the other half ($0\le a\le 1$), for some $a>0$. Assume that there are an even number of atoms. (Hint: think of this as a multinomial distribution.)%Under the alternative, what is the asymptotic distribution of the statistic $U$?
	\begin{enumerate}
		\item What are the mean and variance of this statistic under the null?
		\item What are the mean and variance of this under the alternative?
		\item What is the asymptotic distribution of $U$ under the null hypothesis that $p_i=1/m$? \textit{Hint: you can use the fact that for $X_1,\dots, X_N\stackrel{i.i.d}{\sim} multinomial(q_1,\dots,q_k)$, $\sum_{i=1}^k (N_i-Nq_i)^2/Nq_i\stackrel{d}{\rightarrow} \chi^2_{k-1}$, where $N_i$ is the number of datapoints with value $i$.}
		\item Under the alternative hypothesis,is it always the case that $U$ has a limiting normal distribution? Can you give a sufficient condition on the number of atoms $m$  so that this is true?  
		\textit{Hint: Your variance will have two parts, and when the first one (with $1/n$ dependence on $n$) dominates the second (with $1/n^2$ dependence on $n$), you have a normal convergence. Typically, if $m$ is small, the first one will dominate, however, it is possible that $m$ is very large, in so you need $n$ to be sufficiently large for the first term to dominate the second. }
		%\item Write down the probability of accepting the null hypothesis (which is $p_i=1/m$, for all $i$), when in fact the underlying distribution if coming from the non-uniform distribution parametrized by $a$ described above.
		%\item How big does $n$ and $a$ have to be so that the above probability to be smaller than some small fraction $\delta$? To be concrete, provide a lower bound on $n$ in terms of $m,\ \delta$ and $\epsilon$.
	\end{enumerate}
\end{enumerate}
 \item (VC dimension) Compute the VC dimension of the following function classes. You can take it as everything on or inside the shape is +ve.
\begin{enumerate}
	\item Circles in $\R^2$ 
	\item Axis aligned rectangles in $\R^2$
	\item Axis aligned squares in $\R^2$
\end{enumerate}
\item We will find the covering number of ellipses in this problem.  Given a collection of positive numbers $\{\mu_j,j=1\dots d\}$, consider the ellipse \begin{align}\label{eq:ell}
\mathcal{E}=\{\theta\in\R^d : \sum_i \theta_i^2/\mu_i^2\leq 1\}
\end{align}
\begin{enumerate}
	\item Show that $$\log N(\epsilon; \mathcal{E},\|.\|_2)\geq d\log (1/\epsilon)+\sum_{j=1}^d\log \mu_j$$
	\textit{Hint: you can use the fact that the volume of the Ellipse defined in Eq~\ref{eq:ell} is given by $\prod_{i=1}^d \mu_i \times C_d$ where $C_d$ is the volume of an unit sphere in $d$ dimensions. Extra credit for proving this! All you have to do is a simple substitution!}
	\item Now consider an infinite-dimensional ellipse, specified by the sequence $\mu_j = j^{-2\beta}$
	for some parameter $\beta > 1/2$. Show that
	$$\log N(\epsilon; \mathcal{E},\|.\|_2)\geq C \left(\frac{1}{\epsilon}\right)^{1/2\beta},$$
	where $\|\theta-\theta'\|_{\ell_2}^2=\sum_{j=1}^\infty (\theta_i-\theta'_i)^2$ is the squared $\ell_2$-norm on the space of square summable sequences.
	\textit{Hint: This is going to involve a truncation argument. Truncate at $d$ dimension, and obtain a relationship of the original covering number with the covering number of the truncated ellipse. Use your earlier result for $d$ and then optimize over $d$. }
\end{enumerate}
\end{enumerate}
\end{document} 
