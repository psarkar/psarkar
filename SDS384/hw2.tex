\documentclass[11pt]{article}
\usepackage[left=1.25in,top=1in,right=1.25in,bottom=1.00in]{geometry}
\usepackage{amsmath,amssymb}
\usepackage{amsbsy}
\usepackage{amsthm}
\usepackage{hyperref}

\usepackage{epsfig}
\usepackage{color}
\usepackage[round]{natbib}
\usepackage{multirow} 
\newcommand{\logit}{\mbox{logit}}
\newcommand{\probit}{\mbox{probit}}
\newcommand{\hiw}{{\small\textsc{HIW}}}
\newcommand{\iw}{{\small\textsc{IW}}}
\newcommand{\N}{\mbox{N}}
\newcommand{\Be}{\mbox{Be}}
\newcommand{\dd}{\mbox{d}}
\newcommand{\C}{\; | \;}
\newcommand{\var}{\text{var}}

\newtheorem{theorem}{Theorem}[section]
\newtheorem{lemma}[theorem]{Lemma}
\newtheorem{proposition}[theorem]{Proposition}
\newtheorem{corollary}[theorem]{Corollary}

\newenvironment{definition}[1][Definition]{\begin{trivlist}
\item[\hskip \labelsep {\bfseries #1}]}{\end{trivlist}}
\newenvironment{example}[1][Example]{\begin{trivlist}
\item[\hskip \labelsep {\bfseries #1}]}{\end{trivlist}}
\newenvironment{remark}[1][Remark]{\begin{trivlist}
\item[\hskip \labelsep {\bfseries #1}]}{\end{trivlist}}




\begin{document}

\title{{\bf Homework Assignment 2}\\Due Friday March 4th midnight}
\author{SDS 384-11 Theoretical Statistics}

\date{}

\maketitle{}
%\textbf{Set algebra and probability laws.}
\begin{enumerate}%\item Read Bertsekas and Tsitsiklis, sections 1.1, 1.2 and 1.3.
%\section{Jeffrey's prior}
%\begin{enumerate}
%\end{enumerate}


\item Remember Hoeffding's Lemma? We proved it with a weaker constant in class using a symmetrization type argument. Now we will prove the original version. Let $X$ be a bounded r.v. in $[a,b]$ such that $E[X]=\mu$.
Let $f(\lambda)=\log E[e^{\lambda(X-\mu)}]$. Show that $f''(\lambda)\leq (b-a)^2/4$. Now use the fundamental theorem of calculus to write $f(\lambda)$ in terms of $f''(\lambda)$ and finish the argument.
\item Consider a r.v. $X$ such that for all $\lambda\in \Re$
\begin{align}
E[e^{\lambda X}]\leq e^{\frac{\lambda^2\sigma^2}{2}+\lambda\mu}
\end{align}
Prove that:
\begin{enumerate}
	\item $E[X]=\mu$.
	\item $\var(X)\leq \sigma^2$.
	\item If the smallest value of $\sigma$ satisfying the above equation is chosen, is it true that $\var(X)=\sigma^2$? Prove or give a counter example.
\end{enumerate}
\item Given a positive semidefinite matrix $Q\in \mathbb{R}^{n\times n}$, consider $Z=\sum_{i,j}Q_{ij}X_iX_j$. When $X_i\sim N(0,1)$, prove the Hanson-Wright inequality.
$$P\left(Z\ge \text{trace}(Q)+ t\right)\leq \exp\left(-\min\left\{c_1 t/\|Q\|_{op},c_2 t^2/\|Q\|_F^2\right\}\right),$$
where $\|Q\|_{op}$ and $\|Q\|_{F}$ denote the operator and frobenius norms respectively. \textit{Hint: The rotation-invariance of the
 Gaussian distribution and sub-exponential nature of $\chi^2$-variables could be useful.}
\item We will prove properties of subgaussian random variables here. Prove that:
\begin{enumerate}
\item Moments of a  mean zero subgaussian r.v. $X$ with variance proxy $\sigma^2$ satisfy:
\begin{align}
E[|X^{k}|]\leq k2^{k/2}\sigma^k \Gamma(k/2),
\end{align}
where $\Gamma$ is the gamma function. 
\item If $X$ is a mean 0 subgaussian r.v. with variance proxy $\sigma^2$, prove that,
$X^2-E[X^2]$ is a subexponential $(c_1\sigma^2,c_2\sigma^2)$ (we are using the $(\nu,b)$ parametrization of subexponentials we did in class, so $\nu^2$ is the variance proxy). Here $c_1,c_2$ are positive constants.
\item Consider two independent mean zero subgaussian r.v.s $X_1$ and $X_2$ with variance proxies $\sigma_1^2$ and $\sigma_2^2$ respectively. Show that $X_1X_2$ is a subexponential r.v. with parameters $(d_1\sigma_1\sigma_2, d_2\sigma_1\sigma_2)$. Here $d_1,d_2$ are positive constants.
\end{enumerate}
\end{enumerate}
\end{document} 
